\documentclass[twocolumn]{article}
\usepackage[fleqn]{mathtools}%for aligning parameters of fitted erfc(x)
\usepackage{geometry}	%
\usepackage{abstract} %to get email footnotes
\geometry{margin=2cm}	%more visible figures (more place) 
\usepackage[superscript,biblabel]{cite}%superscript citing
\usepackage[utf8]{inputenc}
\usepackage[english]{babel}
\usepackage{amsmath}	%booklet
\usepackage{hyperref}	%clickable citings, referencing URL via \url{}
\usepackage{siunitx}	%for SI units; see ftp://ftp.dante.de/tex-archive/macros/latex/exptl/siunitx/siunitx.pdf
\usepackage{graphicx} 	%includegraphics
\usepackage{mhchem}		%writing chemical elements with mass numbers
\usepackage[nottoc]{tocbibind}	%references
\usepackage{indentfirst}%indenting first paragraphs
\newcommand{\insertFigure}[1]{%
   \includegraphics[width=0.95\linewidth]{#1}%
}
\title{\textbf{detuning}}
\author{Bence Mitlasóczki\thanks{s6bemitl@uni-bonn.de} and Beno\^it Scholtes\thanks{s6bescho@uni-bonn.de} \\ \textit{Rheinische-Friedrich-Wilhelms Universit\"at Bonn}}
\begin{document}
\section{new things}
\subsection{Polarization direction dependence}
As it is visible on Figure \ref{fig:Waveplate}, the MOT luminosity changes periodically, following a $\pi$ periodic dependence; the fitted function
\begin{equation}
f(x) = A \cos\big(\frac{\pi x}{180^{\circ}} - \varphi \big) + B \nonumber
\end{equation}
serves the purpose of confirming the periodicity, as the fitting is otherwise distorted by the abundance of near-maximum points compared to those near the minimum.\\
Figure \ref{fig:Waveplate2} shows the unadjusted luminosity as a function of the polarization angle of the beam changed before the second passing through the sample. Here, the strong fluctuations make it impossible to notice fine angle-dependent changes. The underlying function is likely a constant function, shown as the red dashed fitted line.
\begin{figure}
\centering
\insertFigure{Images/Waveplate1_w_fit.png}
\caption{MOT luminescence intensity with background as a function of the polarization angle before the first MOT pass-through of one beam.}
\label{fig:Waveplate}
\end{figure}
\begin{figure}
\centering
\insertFigure{Images/Waveplate2.png}
\caption{MOT luminescence intensity with background as a function of the polarization angle before the second MOT pass-through of one beam.}
\label{fig:Waveplate2}
\end{figure}
\subsection{Detuning of the cooling laser}
To quantify the ideal detuning of the cooling laser, we used a slow periodic signal to scan through the spectrum slow enough that the changes in the MOT can be recorded on an oscilloscope. The detected peaks (F $\rightarrow$ F1 and MOT fluorescence peak):
\begin{alignat*}{3}
&3 \rightarrow 2, \, 4&&: \hspace{12pt} &(4.708 \pm 0.024) \, \text{s}\\
&3 \rightarrow 3, \, 4&&:  &(4.956 \pm 0.024) \, \text{s}\\
&3 \rightarrow 4&&:		 &(5.556 \pm 0.016)\, \text{s}\\
&MOT			&&:		&(5.456 \pm 0.008) \, \text{s}
\end{alignat*}
By matching the difference of peaks 1, 2 and peaks 2, 3 with the frequencies of the transitions, averaging the results yields a frequency scale of
\begin{equation}
C =(114.161 \pm 11.163)\, \frac{\text{MHz}}{\text{s}} \nonumber
\end{equation}
The optimal detuning is the difference between the MOT and the 3 $\rightarrow$ 4 peaks:
\begin{equation}
\Delta = C \cdot (0.100 \pm 0.008) \, \text{s} = (11.4161 \pm 1.4423) \, \text{MHz} \nonumber
\end{equation}
\begin{figure}
\centering
\insertFigure{Images/Detuning_cooling.png}
\caption{MOT luminescence intensity with background as a function of the polarization angle before the second MOT pass-through of one beam.}
\label{fig:Detuning}
\end{figure}
%TODO look at the matrices in nicola bardia
%TODO detuning of repumping laser?


\begin{thebibliography}{9}
\bibitem{manual}
Unspecified Author, \textsl{FP Experiment: Rubidium MOT} (University of Bonn, 2014).
\end{thebibliography}
\end{document}