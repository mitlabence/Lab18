\documentclass{article}
\usepackage{amsmath}
\title{E214 questions}
\begin{document}
\maketitle
\section{Section 7.2.2}
Question A
\begin{equation}
Z^0 \, \rightarrow \, e^+ e^-
\end{equation}
\begin{align}
p_{Z^0} &= \begin{bmatrix}
           m_{Z^0} \\
          \vec{0} \\
         \end{bmatrix} 
         =
         \begin{bmatrix}
         E_e\\
         \vec{p}
         \end{bmatrix}
         +
         \begin{bmatrix}
         E_e\\
         -\vec{p}
         \end{bmatrix}
\end{align}
\begin{equation}
m_{Z^0} = 2 \sqrt{p^2 + m_e^2}
\end{equation}
\begin{equation}
m_{Z^0} = 91.2 \, \text{GeV/$c^2$}, \hspace{6pt} m_{e} = 511 \, \text{GeV/$c^2$}
\end{equation}
\begin{equation}
p = \sqrt{\frac{m_{Z^0}^2}{4} - m_e^2} = 45.6 \, \text{GeV/$c$}
\end{equation}
Question B
\begin{align}
E_{CMS} = 5 \, \text{GeV} \, \Rightarrow \, s = (p_{\tau^+} + p_{\tau^-})^2 &= \begin{bmatrix}
2E_{\tau}\\
\vec{0}
\end{bmatrix}^2 = 4 E_{\tau}^2 = E_{CMS}^2
\end{align}
\begin{equation}
p_{\tau}^2 = E_{\tau}^2 - m_{tau}^2 = \Big( \frac{25}{4} - 1.78^2 \Big) \, \text{GeV$^2$/$c^2$}
\end{equation}
\begin{equation}
p = 1.755 \, \text{GeV/$c$}
\end{equation}
\section{Section 7.4.1}
First question: ptw.
\par We are looking at $W \, \rightarrow e \nu$ processes. We can use the missing transverse momentum (\texttt{ptmis\_x, ptmis\_y}), which we assign to the neutrino. As we know the electron transverse momentum: \texttt{el\_px, el\_py, el\_pt}, the W-boson transverse momentum is determined by momentum conservation.\\[14pt]
Fitting.
\par Also see Barlow pp. 58-60 (4.11-12): the standard deviation squared of a function $f(x,y)$ is given by
\begin{equation}
\sigma_f^2 = \Big(\frac{df}{dx}\Big)^2 \sigma_x^2 + \Big( \frac{df}{dy} \Big)^2 \sigma_y^2 + 2 \frac{df}{dx}\frac{df}{dy}\text{cov}(x,y)
\end{equation}
\textbf{Minimize correlation: no clue. Correlation is maximal if the variables are linearly dependent ($\pm 1$), 0 if they are statistically independent.}
\section{Section 7.5.1}
Minimum invariant 4-lepton mass
\par At threshold: $m_{Z^0} = 2 m_l$; the minimum 4-lepton invariant mass is acquired when both $Z^0$'s are stationary (CMS frame),and equal to $2m_{Z^0}$.\\[14pt]
Lepton mass distribution
\par See manual p. 41, also Thomson 17.19; there is a $Z^0$ peak and a Higgs-boson peak. One example for the background at the Higgs-peak is the $t \rightarrow bW^+$ decay.\\[14pt]
Ideal/real detector.
\par Ideal: there would be no missing transverse momentum. In real detectors, there are energy losses not detected due to inactive detector elements, imperfect calibration (?).\\[14pt]
Four leptons in $t \bar{t}$ event.
\par Might be manual p. 46 Figure 6.2?\\[14pt]
Bins.
\par The error for bin entries is (counting statistics) $\sqrt{100} = 10$. Finding a bin with 130 entries (or more): $\sigma = 10 \rightarrow 3\sigma$ is what we are looking for. The probability of higher than 130 or lower than 70 is $ 1- 3 \sigma = 1-0.9973$, so the requested probability for 1 bin is $(1-3\sigma)/2$. The probability that at least one of the bins has more than 130 counts:
\begin{equation}
P = 1 - \Big( 1 - \frac{1- 0.9973}{2} \Big)^{200} = 1-0.7532 = 0.2368.
\end{equation} 
\end{document}