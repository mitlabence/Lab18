\documentclass[twocolumn]{article}
\usepackage{geometry}	%
\usepackage{abstract} %to get email footnotes
\geometry{margin=2cm}	%more visible figures (more place) 
\usepackage[superscript,biblabel]{cite}%superscript citing
\usepackage[utf8]{inputenc}
\usepackage[english]{babel}
\usepackage{amsmath}	%booklet
\usepackage{hyperref}	%clickable citings, referencing URL via \url{}
\usepackage{siunitx}	%for SI units; see ftp://ftp.dante.de/tex-archive/macros/latex/exptl/siunitx/siunitx.pdf
\usepackage{graphicx} 	%includegraphics
\usepackage{mhchem}		%writing chemical elements with mass numbers
\usepackage[nottoc]{tocbibind}	%references
\usepackage{indentfirst}%indenting first paragraphs

%the command \insertFigure{file} inserts figure with width 0.9*(column width)
\newcommand{\insertFigure}[1]{%
   \includegraphics[width=0.95\linewidth]{#1}%
}

\title{\textbf{E212: Properties of Elementary Particles}}
\author{Bence Mitlasóczki\thanks{s6bemitl@uni-bonn.de} and Beno\^it Scholtes\thanks{s6bescho@uni-bonn.de} \\ \textit{Rheinische-Friedrich-Wilhelms Universit\"at Bonn}}
\begin{document}
\renewcommand{\abstractname}{\vspace{-\baselineskip}} %supresses abstract title
\twocolumn[ %makes a one column abstract
\begin{@twocolumnfalse}
\maketitle
\begin{abstract} \vspace{-8mm}
Abstract goes here
\end{abstract}
\end{@twocolumnfalse}
\hspace{5mm} ]
\maketitle
\saythanks %from abstract package to ensure email footnotes from \thanks command in a two-collumn article
\section{Introduction}
Introduction text

\section{Theory}
Our current best understanding of particle physics is the Standard Model of particle physics (SM). This model describes our discoveries of fundamental particles and their interactions with each other, mediated by fundamental forces. It furthermore describes the way these particles and forces combine to form atoms, from which many of the physical phenomena we encounter in everyday life can be explained. The main shortcoming of the SM however, is its inability to be united with gravity. That said, due to the relative weakness of gravity in comparison to the other fundamental forces, it rarely has an effect in particle physics and thus is mainly ignored.

\subsection{The Standard Model}
Figure~\ref{fig:part} give a summary of the particles in the SM with their most basic properties. Furthermore, there also exists anti-particles of many of these particles. An anti-particle, such as a positron, is identical to its particle (electron) apart from having the opposite electric charge. A neutral particle is often its own anti-particle such as the Z~boson, though neutrinos have anti-particles which are merely distinct by having opposing spin projections. All matter particles and the W~bosons have anti-particles while the rest are their own anti-particles.
\begin{figure}[!h]
	\centering
	\insertFigure{SM.png}
	\caption{Illustration of the elementary particles in the SM. Quarks are in purple and Leptons in green, arranged into generation columns from left to right. The gauge bosons are in red, with the scalar Higgs boson in yellow.~\cite{part}}
	\label{fig:part}
\end{figure}
Quarks and leptons make up all the matter particles that have been discovered. These are given in three generations of particles, shown with the columns from left to right in the figure. Matter that is encountered everyday is largely structured from the first generation, namely the electron, electron neutrino, up quark, and down quark. For example, atoms are made up of electrons, protons, and neutrons, the latter two being composed of up and down quarks. The second and third generations are composed of particles which are otherwise exactly identical to their first generation counterpart apart from being heavier, the third generation being the heaviest of them all. This is only known to be true for the charged leptons (electron, muon, and tau) and quarks however. Though the neutrinos are known not to be massless, they have very small masses which have not been accurately measured. It is unknown which is the most massive and which the least.~\cite{Thompson} Figure~\ref{fig:mass} illustrates the relative masses of the matter particles.
\begin{figure}[!h]
	\centering
	\insertFigure{mass.png}
	\caption{Illustration of the relative masses of the matter particles in their respective generations. The neutrinos are left blank to show that their masses are extremely small in comparison the other particles.~\cite{Thompson}}
	\label{fig:mass}
\end{figure}
The main reason why the second and third generation of particles are largely not existent in everyday phenomena is due to the requirement that higher energies are needed to produce these heavier particles. Furthermore, these heavier particles have shorter lifetimes due to their favourable decay into lighter particles, such as those in the first generation, due to the fundamental tendency of physical systems to higher kinetic energy states. Particle physics experiments need to be performed at increasingly higher energies in order to produce more massive particles that we do not readily observe. This is illustrated in Figure~\ref{fig:energy}. It should be noted that though neutrinos are not seen, trillions of solar neutrinos pass through your body each second, oscillating between their three different flavours.~\cite{Thompson} They are extremely difficult to detect due to the fact that they have a small mass and no electric charge. \\
\begin{figure}[!h]
	\centering
	\insertFigure{energy.png}
	\caption{Illustration of the energies required to probe different structures and particles.~\cite{Thompson}}
	\label{fig:energy}
\end{figure}
\par Figure~\ref{fig:part} also shows the fundamental forces in the SM which are all mediated via the exchange of a gauge boson. The most familiar of these is the photon $\gamma$ which mediates the electromagnetic force, responsible for electricity, magnetism, and light. The photon interacts with all particles that have an electric charge and thus with all matter particles in the SM apart from the neutrinos. It also interacts with the W~bosons as they are electrically charged. The gluon gauge bosons mediate the strong force, thus called as it is the strongest force and binds nuclei and hadrons together, explained in Section~\ref{sec:hadrons}. The gluons interacts only with particles which have a so-called "colour" charge, another property of particles similar to electric charge. Only quarks have a colour charge and the eight differently coloured gluons. Next, the oppositely charged W$^{\pm}$ and Z~bosons mediate the weak interaction, the weakest force apart from gravity. This force is responsible for radioactive decay and interacts with all matter particles in the SM. Finally, the Higgs boson is the most recently discovered particle which gives mass to all the matter particles in the SM by interacting with them.

\subsection{Hadrons and the Strong Force}
Though quarks are elementary in the SM, they cannot be observed as free particles. This is because quantum chromodynamics (QCD) of the SM, the theory of the strong force, states that colour is confined such that systems with a colour charge cannot propagate freely. Instead, only colourless composite particles can be observed, termed "colour confinement". As a result, quarks form composite particles called hadrons which are bound by gluons. Hadrons generally form baryons, composed of three quarks, and mesons, composed of one quark and one anti-quark. The reason for the these two types is a result of there being three colours, red, blue, and green, and three anti-colours, anti-red, anti-blue, and anti-green. Colourlessness is achieved by combining all three colours (or anti-colours) in a baryon, or a colour and its anti-colour in a meson, as shown in Figure~\ref{fig:colour}. Protons and neutrons are examples of baryons, composed of two up quarks and one down quark (written as uud), and two down quarks and one up quark (udd), respectively.
\begin{figure}[!h]
	\centering
	\insertFigure{colour.png}
	\caption{Colour confinement resulting in baryons and mesons, given with some examples.~\cite{colour}}
	\label{fig:colour}
\end{figure}
Furthermore, exotic hadrons, which achieve colour confinement with more quarks, have been hypothesised and observed, though without explicit confirmation that the observations were indeed bound exotic hadrons. Examples include the tetraquark with two quarks and two respective anti-quarks, and the pentaquark with four quarks and an anti-quark. Exotic hadrons are rare however, due to the tendency of quarks to form and decay quickly into mesons and baryons.

\subsection{Multiplets}
The quark model of QCD just explained was originally proposed from a group theoretic perspective by Murray Gell-Mann among others in the 1960s. This method has proven to be extremely accurate at predicting particles and their properties, explaining all the hadrons that have been observed. It predicts all the possibilities of obtaining colourless composite particles from the quarks in the SM, as well as the colours and the anti-colours. Hadrons are often grouped into so called "multiplets" in this framework by their composite particles. Figure~\ref{fig:pseudo} shows one such multiplet of mesons formed from up, down, and strange quarks, as well as their anti-quarks. Noteworthy is the strangeness quantum number, shown in the figure, defined as $S = n_{\bar{s}} - n_s$, where $n_{\bar{s}}$ and $n_s$ are the number of strange anti-quarks and quarks in the hadron, respectively~\cite{Thompson}. Strangeness has been observed to be conserved in strong and electromagnetic interactions, but not in weak interactions.
\begin{figure}[!h]
	\centering
	\insertFigure{pseudo.png}
	\caption{The light pseudoscalar meson nonet, plotted against strangeness and electric charge axes.~\cite{pseudo}}
	\label{fig:pseudo}
\end{figure}
The pions are of particular importance of this experiment.
%TODO give pion details and Baryon decuplet


\subsection{Bubble Chamber}


\iffalse
\begin{figure}[!h]
\centering
%\insertFigure{}
\caption{A sample figure}
\label{fig:example}
\end{figure}
\fi
The number of protons passing through the bubble chamber (in $x$ direction) obeys
\begin{equation}
N(x) = N_0 e^{-n \sigma x}
\end{equation}
with $N_0$ being the initial proton number, $n$ the number density, $\sigma$ the (total) cross-section.
\section{Experimental Method} \label{sec:Exp}



\section{Procedure} \label{sec:Proc}
\subsection{Magnification}
We determined the magnification of the photographs by comparing the known coordinates of marks on the two glass planes with the measured distances and assuming the beam passes through the middle of the bubble chamber.\cite{seul}
\begin{table}
\centering
\begin{tabular}{|c|c|c|}
\hline
Distance (cm):	&	F21	- F22	&	G41	- G42\\\hline
calculated	&	23.9951		&  32.1905\\\hline
measured	&	$28.2 \pm 0.1$		&  $37.7 \pm 0.1$	\\\hline
$V_{\text{F}}$, $V_{\text{G}}$	& $0.85089 \pm 0.00302$	& $0.85386 \pm 0.00226$\\\cline{1-3}
$V_{a}$	& $0.85238 \pm 	0.00264$ \\
\cline{1-2}
\end{tabular}
\end{table}
To get the true depth at which the beams were passing through, we used the "stereo-shift" method in 23 different cases. Viewing the same event from two different cameras, we measured the displacement $s_G$ of the point G41 and $s_A$ of an easily identifiable event in the path of the beam, both with an error of $\pm 0.1$~cm. From the data gathered, we discovered the depth to be at
\begin{equation}
\frac{s_A}{s_B} = 0.570 \pm 0.011,
\end{equation}
of the total depth, which is in disagreement with our assumption for the magnification before, namely that the beam passes through at $0.5$ depth. This is an important source of systematical error when measuring length on the photo and reconstructing real distances from it.
\subsection{Scattering events}
For our next task, we analyzed 50 records, identifying and counting elastic and inelastic scatterings between the marks F21 and F35. Of the 532 total incoming protons, we found 27 of them interacted with the hydrogen in the chamber via elastic, and 64 inelastic scattering.
\section{Conclusion}

\begin{thebibliography}{9}
\bibitem{Thompson}
M. Thompson, \textsl{Modern Particle Physics} (Cambridge University Press, New York, 2013).
\bibitem{booklet}
Unspecified author, \textsl{Advanced Laboratory Course (physics601): Description of Experiments} (University of Bonn, 2018).
\bibitem{leo}
W. R. Leo, \textsl{Techniques for Nuclear and Particle Physics Experiments} (Springer-Verlag, 1987), p. 305.
%William R. Leo
\bibitem{seul}
G. Seul, \textsl{Properties of elementary particles} (University of Bonn, 2009).
\bibitem{part}
T. DeMichele, \textit{The Standard Model (of Particle Physics) Explained}, WWW Document, \url{http://factmyth.com/the-standard-model-of-particle-physics-explained/}.
\bibitem{colour}
H. Klus, \textit{The Strong Nuclear Force}, WWW Document, \url{http://www.thestargarden.co.uk/Strong-nuclear-force.html}.
\bibitem{pseudo}
J. Denker, \textit{Quarks -\textgreater Mesons -\textgreater Nonet = Octet plus Singlet}, WWW Document, \url{https://www.av8n.com/physics/quark-meson-nonet.htm}.

\iffalse
\bibitem{book}
K. Siegbahn, \textsl{Alpha-, beta-, and gamma-ray spectroscopy, Vol. 2} (North Holland Publishing Company, Amsterdam, 1965).
 \bibitem{link}
 W. U. Boeglin, \textit{Scintillation Detectors}, WWW Document, \url{http://wanda.fiu.edu/teaching/courses/Modern_lab_manual/scintillator.html}.
\bibitem{pdf_on_website}
Unspecified author, \textsl{Gamma Ray Spectroscopy} (University of Florida, 2013), \url{https://www.phys.ufl.edu/courses/phy4803L/group_I/gamma_spec/gamspec.pdf}.
\bibitem{cfd}
E. Ermis and C. Celiktas, International Journal Of Instrumentation Science 1, (2013), pp.54-62.
%alternative url: https://en.wikipedia.org/wiki/Constant_fraction_discriminator
\bibitem{signal}
M. Nakhostin, \textsl{Signal Processing for Radiation Detectors} (John Wiley $\&$ Sons, 2018), p. 298\footnote{Relevant pages (chapter 6) available for preview under\\ \url{https://books.google.de/books?id=Lrg4DwAAQBAJ}}.
%Mohammad Nakhostin
\bibitem{meliss}
A. C. Melissinos, J. Napolitano, \textsl{Experiments in Modern Physics, 2\textsuperscript{nd} edition} (Academic Press, San Diego, 2003), pp 419-21.
\fi
\end{thebibliography}
\end{document}