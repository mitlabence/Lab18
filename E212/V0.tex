\documentclass[twocolumn]{article}
%TODO copy this into E212
\usepackage{multirow}

\usepackage{geometry}	%
\usepackage{abstract} %to get email footnotes
\geometry{margin=2cm}	%more visible figures (more place) 
\usepackage[superscript,biblabel]{cite}%superscript citing
\usepackage[utf8]{inputenc}
\usepackage[english]{babel}
\usepackage{amsmath}	%booklet
\usepackage{hyperref}	%clickable citings, referencing URL via \url{}
\usepackage{siunitx}	%for SI units; see ftp://ftp.dante.de/tex-archive/macros/latex/exptl/siunitx/siunitx.pdf
\usepackage{graphicx} 	%includegraphics
\usepackage{mhchem}		%writing chemical elements with mass numbers
\usepackage[nottoc]{tocbibind}	%references
\usepackage{indentfirst}%indenting first paragraphs

%the command \insertFigure{file} inserts figure with width 0.9*(column width)
\newcommand{\insertFigure}[1]{%
   \includegraphics[width=0.95\linewidth]{#1}%
}

\title{\textbf{E212: Properties of Elementary Particles}}
\author{Bence Mitlasóczki\thanks{s6bemitl@uni-bonn.de} and Beno\^it Scholtes\thanks{s6bescho@uni-bonn.de} \\ \textit{Rheinische-Friedrich-Wilhelms Universit\"at Bonn}}
\begin{document}
\section{Theory}
Useful formula:
\begin{equation}\label{eqn:radius}
p = 0.3 \cdot B \cdot V \cdot r
\end{equation}
\section{Results}
\subsection{Neutrino momentum}
The pion had an initial radius of $27.0 \pm 0.9$~cm, this gives a momentum of $120.1 \pm 3.8$~MeV/c. The track length was measured to be $79.3 \pm 0.9$~cm. The graph provided gave a corresponding $128.5 \pm 5.9$~MeV/c. Comparing the two values, we infer that the pion has indeed decayed while at rest in the laboratory frame.\\
Next, we measured the length of the $\mu$ track, and found it to be $0.597 \pm 0.085$~cm. From the graph again, a momentum of $27.64 \pm 1.26$~MeV/c was read, in fairly good agreement with the theoretical $29.8$~MeV/c value.
\subsection{V0}
We found two V$_0$-canditate events. It is important to note that the angle between the two produced particles in each case was close to 0$^{\circ}$, so the possibility of these being pair productions is considerable.\\
On image 2898 (???) we detected a primary vertex with 2 visible outgoing particles and one distant vertex of two particles with opposite charges (meaning it was a decay process) which is suspected to have come from the primary vertex. The distance of the two vertices was measured to be 137~cm. 
\subsubsection{The secondary vertex}
The neutral particle decayed into two particles with an angle of $(0\pm 0.1)^{\circ} $ between them, this made the association to the primary vertex an easy task. We measured the two radii to be $(56 \pm 2)$~cm for the negative, $750 \pm 50$ for the positive particle. From Eqn. \ref{eqn:radius}, in a coordinate system with the x-axis along the supposed V$_0$ path, we get 
\begin{alignat*}{3}
p_- &= &&((249.2 \pm 8.9)\, \text{MeV/c},\hspace{3pt} && (0.22 \pm 0.22)\, \text{MeV/c})\\
p_+ &= &&((3337.1 \pm 222.7)\, \text{MeV/c},\hspace{3pt} && (-2.92 \pm 2.92)\, \text{MeV/c})
\end{alignat*}
The total V$_0$ momentum is then
\begin{equation}
 \vert p_0 \vert = (3586.21 \pm 222.89)\, \text{MeV/c}
\end{equation}
\subsubsection{A first look at the primary vertex}
The primary vertex consists of the incoming proton, and two positively charged particles: particle 1 has a path with radius $(2000 \pm 200)$~cm and angle $(2 \pm 0.3)^{\circ}$, particle 2 $(1700 \pm 100)$~cm and $(4.5 \pm 0.3)^{\circ}$. Using the incoming beam as the direction of the x-axis and the positive quarter plane being the top right one, we can write down the momenta:
\begin{alignat*}{3}
p_1 &= &&((8893.4 \pm 889.8)\, \text{MeV/c},\hspace{3pt} && (310.6 \pm 56.0)\, \text{MeV/c})\\
p_2 &= &&((7540.7 \pm 444.2)\, \text{MeV/c},\hspace{3pt} && (-593.5 \pm 52.7)\, \text{MeV/c})\\
p_0 &= &&((3585.7 \pm 62.6)\, \text{MeV/c},\hspace{3pt} && (222.9 \pm 19.2)\, \text{MeV/c})\\[6pt]
\Sigma p &= &&((20019.7 \pm 1019.1)\, \text{MeV/c},\hspace{3pt} && (-220.3 \pm 79.3)\, \text{MeV/c})
\end{alignat*}
The momenta of the three particles do not add up to the momentum of the incoming proton (23877 MeV/c in the x-direction), this offers two probable explanations to check first:
\begin{itemize}
\item Another neutral particle was created which was not detected, or
\item One neutral particle was created that decayed into neutral particles very close to the primary vertex, and one of these decayed, showing up as a secondary vertex.
\end{itemize}
Of course other options are also possible, but we are hoping to find the event to be one of these two types.\\
\begin{table}
\centering
\begin{tabular}{|c|c|c|c|c|}
\hline
Name & Mass (MeV) & Distance (m) & Decays into & Fraction \\
\hline
$\Lambda$ 	& 1115.7 & 0.079 &p$\, \pi^-$ & 63.9\%\\
\hline
K$^0_{\text{S}}$	 & 497.6 & 0.027 & $\pi^+ \, \pi^-$ & 69\%\\
\hline
\multirow{2}{*}{K$^0_{\text{L}}$} &  \multirow{2}{*}{ 497.6}& \multirow{2}{*}{15.3}&$\pi^{\pm} \, e^{\mp} \nu_{\text{e}}$ & 40.6\%\\
 & & &$\pi^{\pm} \mu ^{\mp} \nu_{\mu}$&27.0\%\\
\hline
$\Sigma^0$ & 1192.6 & $2.22 \cdot 10^{-11}$& $\Lambda \, \gamma$ & 100\%\\
\hline
$\Xi^0$ & 1314.9 & 0.087 & $\Lambda \, \pi^0$ & 99.5\%\\
\hline
\end{tabular}
\caption{Possible neutral particles\cite{pdg}}
\label{tab:neutral}
\end{table}
\begin{figure}
\centering
\insertFigure{Images/primary_1.png}
\caption{First primary vertex examined.}
\label{fig:primary1}
\end{figure}
\subsubsection{Determining V$_0$}
Unfortunately, we cannot conclude much from the secondary vertex, situated $(138 \pm 1)$~cm away from the primary vertex, showing the path of two particles which leave the chamber. We assume that the V$_0$ particle decayed into two charged particles and nothing else, as the overall momentum already matches what we expect, thus a possible neutral particle would have a pro- or retrograde motion, and we regard this as unlikely. Looking at Table \ref{tab:neutral}, we see 3 possible scenarios:
\begin{itemize}
\item V$_0$ is a $\Lambda$ particle, another neutral particle left the primary vertex undetected; the secondary vertex contains a proton and a pion,
\item instead of a $\Lambda$, a $K^0_{\text{S}}$ was created, which decayed into a pair of pions,
\item A $\Sigma^0$ was created at the proton-proton collision, which then decayed within a few picometers, resulting in a $\Lambda$ baryon which decayed into a proton and a pion, and a photon.
\end{itemize}
We can calculate the mass of the V$_0$ for these scenarios:
\begin{align*}
m_{\text{V}_0} ( \text{p}, \, \pi^-) &= (1103.2 \pm 1028.3)\, \text{MeV}\\
m_{\text{V}_0} ( \pi^+, \, \pi^- ) &= (532.8 \pm 2132.2)\, \text{MeV}
\end{align*}
The uncertainty is tremendous in both cases. We decided for the proton-pion case after comparing the relative errors. This means the V$_0$ was a $\Lambda$ baryon.
\subsubsection{Primary vertex revisited}
The incident proton 
\begin{thebibliography}{9}
\bibitem{booklet}
Unspecified author, \textsl{Advanced Laboratory Course (physics601): Description of Experiments} (University of Bonn, 2018).
\bibitem{leo}
W. R. Leo, \textsl{Techniques for Nuclear and Particle Physics Experiments} (Springer-Verlag, 1987), p. 305.
%William R. Leo
\bibitem{seul}
G. Seul, \textsl{Properties of elementary particles} (University of Bonn, 2009).
\bibitem{pdg}
Particle Data Group.
\iffalse
\bibitem{book}
K. Siegbahn, \textsl{Alpha-, beta-, and gamma-ray spectroscopy, Vol. 2} (North Holland Publishing Company, Amsterdam, 1965).
 \bibitem{link}
 W. U. Boeglin, \textit{Scintillation Detectors}, WWW Document, \url{http://wanda.fiu.edu/teaching/courses/Modern_lab_manual/scintillator.html}.
\bibitem{pdf_on_website}
Unspecified author, \textsl{Gamma Ray Spectroscopy} (University of Florida, 2013), \url{https://www.phys.ufl.edu/courses/phy4803L/group_I/gamma_spec/gamspec.pdf}.
\bibitem{cfd}
E. Ermis and C. Celiktas, International Journal Of Instrumentation Science 1, (2013), pp.54-62.
%alternative url: https://en.wikipedia.org/wiki/Constant_fraction_discriminator
\bibitem{signal}
M. Nakhostin, \textsl{Signal Processing for Radiation Detectors} (John Wiley $\&$ Sons, 2018), p. 298\footnote{Relevant pages (chapter 6) available for preview under\\ \url{https://books.google.de/books?id=Lrg4DwAAQBAJ}}.
%Mohammad Nakhostin
\bibitem{meliss}
A. C. Melissinos, J. Napolitano, \textsl{Experiments in Modern Physics, 2\textsuperscript{nd} edition} (Academic Press, San Diego, 2003), pp 419-21.
\fi
\end{thebibliography}
\end{document}