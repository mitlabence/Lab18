\documentclass[twocolumn]{article}
\usepackage{geometry}	%
\usepackage{abstract} %to get email footnotes
\geometry{margin=2cm}	%more visible figures (more place) 
\usepackage[superscript,biblabel]{cite}%superscript citing
\usepackage[utf8]{inputenc}
\usepackage[english]{babel}
\usepackage{amsmath}	%booklet
\usepackage{hyperref}	%clickable citings, referencing URL via \url{}
\usepackage{siunitx}	%for SI units; see ftp://ftp.dante.de/tex-archive/macros/latex/exptl/siunitx/siunitx.pdf
\usepackage{graphicx} 	%includegraphics
\usepackage{mhchem}		%writing chemical elements with mass numbers
\usepackage[nottoc]{tocbibind}	%references
\usepackage{indentfirst}%indenting first paragraphs

%the command \insertFigure{file} inserts figure with width 0.9*(column width)
\newcommand{\insertFigure}[1]{%
   \includegraphics[width=0.95\linewidth]{#1}%
}

\title{\textbf{Title}}
\author{Bence Mitlasóczki\thanks{s6bemitl@uni-bonn.de} and Beno\^it Scholtes\thanks{s6bescho@uni-bonn.de} \\ \textit{Rheinische-Friedrich-Wilhelms Universit\"at Bonn}}
\begin{document}
\renewcommand{\abstractname}{\vspace{-\baselineskip}} %supresses abstract title
\twocolumn[ %makes a one column abstract
\begin{@twocolumnfalse}
\maketitle
\begin{abstract} \vspace{-8mm}
Abstract goes here
\end{abstract}
\end{@twocolumnfalse}
\hspace{5mm} ]
\maketitle
\saythanks %from abstract package to ensure email footnotes from \thanks command in a two-collumn article
\section{Introduction}
Introduction text

\section{Theory}
Theory
\begin{figure}[!h]
\centering
%\insertFigure{}
\caption{A sample figure}
\label{fig:example}
\end{figure}

\section{Experimental setup} \label{sec:Exp}

\section{Procedure} \label{sec:Proc}
\section{Results}
\section{Conclusion}

\begin{thebibliography}{9}
\bibitem{book}
K. Siegbahn, \textsl{Alpha-, beta-, and gamma-ray spectroscopy, Vol. 2} (North Holland Publishing Company, Amsterdam, 1965).
\bibitem{booklet}
Unspecified author, \textsl{Advanced Laboratory Course (physics601): Description of Experiments} (University of Bonn, 2018).
 \bibitem{link}
 W. U. Boeglin, \textit{Scintillation Detectors}, WWW Document, \url{http://wanda.fiu.edu/teaching/courses/Modern_lab_manual/scintillator.html}.
\bibitem{pdf_on_website}
Unspecified author, \textsl{Gamma Ray Spectroscopy} (University of Florida, 2013), \url{https://www.phys.ufl.edu/courses/phy4803L/group_I/gamma_spec/gamspec.pdf}.
\bibitem{cfd}
E. Ermis and C. Celiktas, International Journal Of Instrumentation Science 1, (2013), pp.54-62.
%alternative url: https://en.wikipedia.org/wiki/Constant_fraction_discriminator
\bibitem{signal}
M. Nakhostin, \textsl{Signal Processing for Radiation Detectors} (John Wiley $\&$ Sons, 2018), p. 298\footnote{Relevant pages (chapter 6) available for preview under\\ \url{https://books.google.de/books?id=Lrg4DwAAQBAJ}}.
%Mohammad Nakhostin
\bibitem{leo}
W. R. Leo, \textsl{Techniques for Nuclear and Particle Physics Experiments} (Springer-Verlag, 1987), p. 305.
%William R. Leo
\bibitem{meliss}
A. C. Melissinos, J. Napolitano, \textsl{Experiments in Modern Physics, 2\textsuperscript{nd} edition} (Academic Press, San Diego, 2003), pp 419-21.
\end{thebibliography}
\end{document}